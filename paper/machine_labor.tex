\documentclass[11pt, a4paper, leqno]{article}
\usepackage{a4wide}
\usepackage[T1]{fontenc}
\usepackage[utf8]{inputenc}
\usepackage{float, afterpage, rotating, graphicx}
\usepackage{epstopdf}
\usepackage{longtable, booktabs, tabularx}
\usepackage{fancyvrb, moreverb, relsize}
\usepackage{eurosym, calc}
% \usepackage{chngcntr}
\usepackage{amsmath, amssymb, amsfonts, amsthm, bm}
\usepackage{caption}
\usepackage{mdwlist}
\usepackage{xfrac}
\usepackage{setspace}
\usepackage[dvipsnames]{xcolor}
\usepackage{subcaption}
\usepackage{minibox}
\usepackage{listings}

\lstdefinelanguage{Stata}{
    keywords={},
    basicstyle=\ttfamily\small,
    commentstyle=\color{gray},
    stringstyle=\color{darkblue},
    showstringspaces=false,
    breaklines=true,
    morestring=[b]",
    morecomment=[l]{///},
    morecomment=[l]{//},
    morecomment=[s]{/*}{*/},
    sensitive=true,
}
% \usepackage{pdf14} % Enable for Manuscriptcentral -- can't handle pdf 1.5
% \usepackage{endfloat} % Enable to move tables / figures to the end. Useful for some
% submissions.

\usepackage[
    natbib=true,
    bibencoding=inputenc,
    bibstyle=authoryear-ibid,
    citestyle=authoryear-comp,
    maxcitenames=3,
    maxbibnames=10,
    useprefix=false,
    sortcites=true,
    backend=biber
]{biblatex}
\AtBeginDocument{\toggletrue{blx@useprefix}}
\AtBeginBibliography{\togglefalse{blx@useprefix}}
\setlength{\bibitemsep}{1.5ex}
\addbibresource{refs.bib}

\usepackage[unicode=true]{hyperref}
\hypersetup{
    colorlinks=true,
    linkcolor=black,
    anchorcolor=black,
    citecolor=NavyBlue,
    filecolor=black,
    menucolor=black,
    runcolor=black,
    urlcolor=NavyBlue
}


\widowpenalty=10000
\clubpenalty=10000

\setlength{\parskip}{1ex}
\setlength{\parindent}{0ex}
\setstretch{1.5}


\begin{document}

\title{Paper Replication: "Seeing beyond the Trees: Using Machine Learning to Estimate the Impact of Minimum Wages on Labor Market Outcomes"\thanks{Lucia Gomez Llactahuamani, University of Bonn. Email: \href{mailto:s6lugome@uni-bonn.de}{\nolinkurl{s6lugome@uni-bonn.de}}.}}

\author{Lucia Gomez Llactahuamani}

\date{\today}

\maketitle


\begin{abstract}

This project replicates the first half of the main results from \citet{cengiz2022seeing}.
Following the authors, I apply machine learning methods to identify the potencial workers 
who are actually affected by the minimum wage policy. Although the code for the second part 
of the paper, estimating the impact of the minimum wage on labor market outcomes, is still 
being developed, this project represents a significant advance. In contrast to the original 
code, which was written in Stata and R, I replicated the study in Python, streamlining all 
the code into a single programming language that is both free and open source. 
This replication project emphasizes the application of best programming practices and 
concepts learned in the EPP course, including functional programming, Pytask, Pytest, and docstrings.
\end{abstract}

\clearpage


\section{Introduction} % (fold)
\label{sec:introduction}

The motivation to replicate the paper of \citet{cengiz2022seeing} comes from the fact that 
they are among the first to apply machine learning techniques to identify the workers 
potentially affected by the minimum wage. While it is easy to locate workers who are 
currently earning the minimum wage, it is difficult to
identify all potential workers who also may have been working had the min-
imum wage been different. This difficulty has led many researchers to focus
on specific industries or demographic groups, such as
teens or younger workers. Their approach of using prediction models 
to classify workers who are likely to be exposed to a minimum wage treatment 
has the main advantage that they are able to 
assess the effect of the minimum wage on a large fraction of low-wage
workers and not just on some specific subgroups with high exposure.

\section{Data} % (fold)
\label{sec:data}

The original data set of the study is stored in 
\url{https://www.dropbox.com/sh/dtjmoo8udmc7ckl/AAD1rz5WalgkwZpiyaoNEtcia?dl=1}. It can be downloaded using the following code:
It can be accessed from Python by using the following code:

\lstset{language=Python}
\lstset{frame=lines}
\lstset{caption={Taking random sample of panel data in Stata}}
\begin{lstlisting}
import requests
from zipfile import ZipFile

url = 'https://www.dropbox.com/sh/dtjmoo8udmc7ckl/AAD1rz5WalgkwZpiyaoNEtcia?dl=1'
zip_file = './src/machine_labor/data/epp_ml_mw_data.zip'
dir_unzipped_files= './bld/python/data'
response = requests.get(url)
with open(zip_file, "wb") as f:
    f.write(response.content)
with ZipFile(zip_file, "r") as zip_ref:
    zip_ref.extractall(path= dir_unzipped_files)
\end{lstlisting}

The primary data used the paper of \citet{cengiz2022seeing} were obtained 
from the 1979-2019 Current Population Survey Outgoing Rotation Group (CPS-ORG), 
which comprises approximately 13 million observations. In order to facilitate 
the replication process, I extracted a random sample of 5\% 
from the total number of observations, resulting in a sample size of 659,641.

The data set I use in this project is stored in 
\url{https://www.dropbox.com/scl/fo/npdgaje0u3ejd5o51ty01/h?dl=1} 
and contains the following data sets:

\begin{itemize}
    \item Current Population Survey Outgoing Rotation Group (CPS-ORG) (cps-morg-2019-new.dta): The 1979-2019 
    CPS-ORG data is a subset of the Basic Montly CPS (CPS-Basic), a montly survey of approximately 60000 households in the United
    States. The CPS-ORG includes only the fourth and eight sample months. It contains data on on weekly earnings 
    and weakly hours worked, as long as, demographic variables, with 659,641 observations and 164 variables\footnote{This 
    data gather information on weekly earnings and weakly hours worked, as long as, demographic variables 
    such as individual's age, race, hispanic status, gender, education, veteran status, marital status, 
    and rural status of residency.}. We consider only 8 of the 164 features for the prediction of the variable
    \texttt{relMW\_groups}: \texttt{age}, \texttt{sex}, \texttt{race}, \texttt{hispanic},
    \texttt{dmarried}, \texttt{educcat}, \texttt{ruralstatus} and \texttt{veteran}. 
    All these variables are relevant when predicting an individual's 
    likelihood of belonging to the group that has a wage close to the minimum. 
    The latest version can be found at this link: \url{https://cps.ipums.org/cps/}
    \item Consumer Price Index (CPI) (cpiursai1977-2019.dta): per year and month for the period 1979-2019. 
    A raw dataset that contains a measure of the average change over time in the prices paid by 
    urban consumers for a market basket of consumer goods and services. 
    The latest version can be found at this link: \url{https://www.bls.gov/cpi/data.htm}
    \item Minimum wage data (VZmw-quarterly-lagsleads-1979-2019.dta):  per state and quarter level. 
    It was built by \citet{vaghul2016historical}.
    The latest version can be found at this link: \url{https://github.com/benzipperer/historicalminwage}
    \item (eventclassification-2019.dta): A dataset with information 
    to identify the relevant post and pre-period around prominent minimum wage changes. It was built by
    the authors of the paper.
\end{itemize}

\section{Predicting who is a minimum wage worker} % (fold)
\label{sec:prediction}

I apply various machine learning methods such as decision tree, random forests, gradient-boosting tree, 
logistic regression and the linear probability model suggested by \citet{card2016myth} to predict who 
is in the group of minimum wahe workers. 

For details on machine learning tree based methods, i.e. decision tree, 
random forests and gradient-boosting tree, review \citet{hastie2009elements}.

Following, I provide the parameters I used in the application of this machine learning techniques.

\begin{itemize}
\item Decision trees
\item Random forest: The parameters the authors use when training this model are
$2,000$ trees and two predictors tried at each split.
\item Boosting: The parameters the authors use are: number of trees $= 4,000$, 
shrinkage factor $= 0.005$, depth of tree $=6$, and minimum observations in a node $=10$.
\item Card and Krueger's linear probability model: Following the authors, I use 
the functional form proposed by \citet{card2016myth}. 
\end{itemize}

In the above models, all numerical features are included linearly, while categorical features 
are expanded into dummy variables. 

In the remaining part I examine who are the most likely minimum wage workers
according to the best-performing prediction model. To compare the 
the models with each other, I employ precision and recall scores. 
\begin{itemize}
    \item "Precision" refers to the share of those I
    classify as being in the predicted group of minimum wage workers who 
    are true minimum wage workers.
    \item "Recall" refers to the share of true minimum wage workers 
    who I correctly classify as being in the predicted group. 
\end{itemize}

The ideal is to construct a predicted group that includes
all the minimum wage workers and none of the non-minimum-wage
workers so that both the precision and the recall are 1. Generally, the higher
the precision for a given recall rate, the better the performance of the model.

Figure 1 shows the precision-recall curves 
corresponding to the various prediction algorithms. 

The figure shows that the decision-tree model outperforms other
prediction models. Most of the machine learning techniques applied 
in the project yielded results similar to those reported in the paper. 
However, the decision tree-model produced a markedly different result, 
which contradicts the literature that suggests it is not among the 
most effective learners. I used the same parameters as the paper, 
but the difference may be attributed to the software package used 
for the calculations. Aside from the decision-tree model, 
the figure shows that the boosted tree model outperforms other
prediction models, since it provides the highest precision at almost all recall
levels. The latest result matches with the findings of the paper.
I are going to taking aside from the analysis the decision-tree model
for the reasons earlier exposed.

\begin{figure}[H]

    \centering
    \includegraphics[width=0.85\textwidth]{../bld/python/figures/precision_recall_curves}

    \caption{\emph{Precision - Recall Curves}}
    \label{fig:precision-recall}

\end{figure}

For comparison, in figure 2 we report 
the other prediction models relative to the boosted tree model. 
The boosted tree model (and also the other
prediction algorithms) improves precision considerably relative to the basic
logistic model. Nevertheless, the differences between the other prediction
models and the boosted tree model are relatively small, especially at higher
recall rate levels. 

\begin{figure}[H]
    \centering
    \includegraphics[width=0.85\textwidth]{../bld/python/figures/precision_relative_boost}
    \caption{\emph{Precision - Recall Curves Relative to Boosting Tree Model}}
    \label{fig:precision-relative}

\end{figure}

The best-performing prediction comes from the gradient-boosting tree
model. The original linear prediction model proposed by Card and Krueger
also performs relatively well, even better than the random forest
and the basic logistic model.

\subsection{Who are the minimum wage workers?}

Once I have selected my best performing prediction model - the boosted tree model -
I examine who are the minimum wage workers by considering  
the relative importance of each predictor. In figure 3 
I plot the "relative influences" of the variables in the gradient-bossting tree model. 
It finds \texttt{age} as the most importantpredictor in the sample with a very large margin.
The variable for educational credentials \texttt{educcat} comes after age. 
Gender variables \texttt{sex} are also relatively important
in the prediction. The indicator variables for \texttt{hispanic}, \texttt{dmarried}, \texttt{race}, and
\texttt{veteran} status appear to have less influence on the prediction.

\begin{figure}[H]

    \centering
    \includegraphics[width=0.85\textwidth]{../bld/python/figures/feature_importance}

    \caption{\emph{Relative influences of the predictors in the boosted tree prediction model}}
    \label{fig:feature_importance}

\end{figure}

By replicating the paper, we obtained similar results, 
which led us to similar conclusions as the authors. 
However, there were slight differences presumably resulting 
from the translation of the codes from Stata and R to Python


\setstretch{1}
\printbibliography
\setstretch{1.5}

% \appendix

% The chngctr package is needed for the following lines.
% \counterwithin{table}{section}
% \counterwithin{figure}{section}

\end{document}
