\documentclass[11pt, a4paper, leqno]{article}
\usepackage{a4wide}
\usepackage[T1]{fontenc}
\usepackage[utf8]{inputenc}
\usepackage{float, afterpage, rotating, graphicx}
\usepackage{epstopdf}
\usepackage{longtable, booktabs, tabularx}
\usepackage{fancyvrb, moreverb, relsize}
\usepackage{eurosym, calc}
% \usepackage{chngcntr}
\usepackage{amsmath, amssymb, amsfonts, amsthm, bm}
\usepackage{caption}
\usepackage{mdwlist}
\usepackage{xfrac}
\usepackage{setspace}
\usepackage[dvipsnames]{xcolor}
\usepackage{subcaption}
\usepackage{minibox}
\usepackage{listings}
% \usepackage{pdf14} % Enable for Manuscriptcentral -- can't handle pdf 1.5
% \usepackage{endfloat} % Enable to move tables / figures to the end. Useful for some
% submissions.

\usepackage[
    natbib=true,
    bibencoding=inputenc,
    bibstyle=authoryear-ibid,
    citestyle=authoryear-comp,
    maxcitenames=3,
    maxbibnames=10,
    useprefix=false,
    sortcites=true,
    backend=biber
]{biblatex}
\AtBeginDocument{\toggletrue{blx@useprefix}}
\AtBeginBibliography{\togglefalse{blx@useprefix}}
\setlength{\bibitemsep}{1.5ex}
\addbibresource{../../paper/refs.bib}

\usepackage[unicode=true]{hyperref}
\hypersetup{
    colorlinks=true,
    linkcolor=black,
    anchorcolor=black,
    citecolor=NavyBlue,
    filecolor=black,
    menucolor=black,
    runcolor=black,
    urlcolor=NavyBlue
}


\widowpenalty=10000
\clubpenalty=10000

\setlength{\parskip}{1ex}
\setlength{\parindent}{0ex}
\setstretch{1.5}


\begin{document}

\title{Paper Replication: "Seeing beyond the Trees: Using Machine Learning to Estimate the Impact of Minimum Wages on Labor Market Outcomes"\thanks{Lucia Gomez Llactahuamani, University of Bonn. Email: \href{mailto:s6lugome@uni-bonn.de}{\nolinkurl{s6lugome@uni-bonn.de}}.}}

\author{Lucia Gomez Llactahuamani}

\date{
    {\bf Preliminary -- please do not quote}
    \\[1ex]
    \today
}

\maketitle


\begin{abstract}
    Some abstract here.
\end{abstract}

\clearpage


\section{Introduction} % (fold)
\label{sec:introduction}

This project replicates the first half of the main results from \citet{cengiz2022seeing} that applies machine learning 
tools to predict who is affected by the policy of minimum wage changes. The code replicating the second part 
of the paper, i.e., implementing an event study using prominent minimum wage increases in the U.S. between 1979
 and 2019, is still ongoing. The original code of the paper is written is Stata and R, the main advantage of 
 replicating it in Python is to unify all the codes in just one programming language that is free and open source. 
 This replication has put emphasis in applying concepts learned in the EPP course such as best programming practices,
 functional programming, Pytask, Pytest and docstrings.


If you are using this template, please cite this item from the references:
\citet{GaudeckerEconProjectTemplates}.


The data set for this project is taken from 
\url{https://www.dropbox.com/sh/dtjmoo8udmc7ckl/AAD1rz5WalgkwZpiyaoNEtcia?dl=0}.
It contains data on smoking habits in the UK, with 1691 observations and 12 variables.
We consider only 4 of the 12 features for the prediction of the variable
\texttt{smoking}: \texttt{marital\_status}, \texttt{highest\_qualification},
\texttt{gender} and \texttt{age}. We model the dependence using a Logistic model. All
numerical features are included linearly, while categorical features are expanded into
dummy variables. Figures below illustrate the model predictions over the lifetime. You
will find one figure and one estimation summary table for each installed programming
language.



\begin{figure}[H]

    \centering
    \includegraphics[width=0.85\textwidth]{../python/figures/smoking_by_marital_status}

    \caption{\emph{Python:} Model predictions of the smoking probability over the
        lifetime. Each colored line represents a case where marital status is fixed to one
        of the values present in the data set.}
    \label{fig:python-predictions}

\end{figure}


\begin{table}[!h]
    \input{../python/tables/estimation_results.tex}
    \caption{\label{tab:python-summary}\emph{Python:} Estimation results of the
        linear Logistic regression.}
\end{table}




% section introduction (end)

\section{Data} % (fold)
\label{sec:data}

The data set for this project is taken from 
\url{https://www.dropbox.com/sh/dtjmoo8udmc7ckl/AAD1rz5WalgkwZpiyaoNEtcia?dl=0}.


It contains data on smoking habits in the UK, with 1691 observations and 12 variables.
We consider only 4 of the 12 features for the prediction of the variable
\texttt{smoking}: \texttt{marital\_status}, \texttt{highest\_qualification},
\texttt{gender} and \texttt{age}. We model the dependence using a Logistic model. All
numerical features are included linearly, while categorical features are expanded into
dummy variables. Figures below illustrate the model predictions over the lifetime. You
will find one figure and one estimation summary table for each installed programming
language.

\lstset{language=Stata}
\lstset{frame=lines}
\lstset{caption={Taking random sample of panel data in Stata}}
\begin{lstlisting}
    use "${data}cps_morg_${endyear}.dta", clear

    tempfile holding
    save `holding'
    keep hhid
    duplicates drop
    set seed 1234
    sample 5 // Draw a 5% sample

    merge 1:m hhid using `holding', keep(match) nogenerate
\end{lstlisting}

\begin{itemize}

    \item Current Population Survey (CPS-ORG)
    \item Consumer Price Index (CPI)
    \item Minimum wage data:  per state and quarter level.
    \item A raw dataset with information 
    to identify the relevant post and pre-period around prominent minimum wage changes. 

\end{itemize}


\section{Prediction Algorithms}
\label{sec: algorithms}

\begin{itemize}
\item Decision trees
\item Random forest
\item Boosting
\item Card and Krueger's linear probability model 
\end{itemize}

\section{Predicting who is a minimum wage worker} % (fold)
\label{sec:prediction}

\subsection{Precision-recall curves and predicted probabilities}

\begin{figure}[H]

    \centering
    \includegraphics[width=0.85\textwidth]{../python/figures/smoking_by_marital_status}

    \caption{\emph{Python:} Model predictions of the smoking probability over the
        lifetime. Each colored line represents a case where marital status is fixed to one
        of the values present in the data set.}
    \label{fig:python-predictions}

\end{figure}

\begin{figure}[H]

    \centering
    \includegraphics[width=0.85\textwidth]{../python/figures/smoking_by_marital_status}

    \caption{\emph{Python:} Model predictions of the smoking probability over the
        lifetime. Each colored line represents a case where marital status is fixed to one
        of the values present in the data set.}
    \label{fig:python-predictions}

\end{figure}

\begin{figure}[H]

    \centering
    \includegraphics[width=0.85\textwidth]{../python/figures/precision_recall_curves}

    \caption{\emph{Precision - Recall Curves:} Plots the precision-recall curves for various
    prediction models described in section IV.A and for a basic logistic model that we
    estimate using (linear) age and categorical education variables. }
    \label{fig:python-predictions}

\end{figure}




\subsection{Who are the minimum wage workers?}


\section{Conclusion}
\label{sec:conclusion}

\setstretch{1}
\printbibliography
\setstretch{1.5}


% \appendix

% The chngctr package is needed for the following lines.
% \counterwithin{table}{section}
% \counterwithin{figure}{section}

\end{document}
